% Условная компиляция для самостоятельной работы
\ifdefined\mainfile
    % Если это часть основного файла, не добавляем начало и конец документа
\else
    \documentclass[12pt, a4paper]{report}
    \usepackage{/Users/vladbelousov/Desktop/Semestr_4-FP-NSU/Настройка/library}
    \usepackage[utf8]{inputenc} % Подключение поддержки UTF-8
    \begin{document}
\fi

%%-------------------------------%%

\[ w(x, t ) = \frac{1 }{\sqrt{1 - 2 xt + t ^2 }} = \sum_{n=0 }^{ \infty  } P_n (x ) t^n  \] 

\[ \frac{\partial  w }{\partial  t } (1 - 2x t +t ^2 ) = (x -t )w   \] 
\[ (1 - 2x t + t ^2 ) \frac{\partial  w }{\partial  t } + (t -x  ) w = 0  \] 
\[ (1 - 2xt + t ^2   ) \sum_{n =0 }^{\infty  } n P_n (x ) t^{ n -1 } + (t -x ) \sum_{n =0} ^{\infty  } P_n (x ) t^n =0   \] 
\[t^{n } : \text{ }   (n+1 ) P_{n+1 } (x )- 2 x nP_n(x ) + (n-\cancel{1} )P_{n-1 } ( x )+ \cancel{P_{n-1 } (x )} - x P_n( x )= 0 ,\text{ }  n \ge 1 \]  
\[ (n +1 )P_{n+1 } (x ) - (2 n +1 ) x P_n( x ) + n P_{n-1 } (x ) = 0 , \text{ } n =1,2, \ldots  \quad \quad (*)\] 

\begin{lemma}
    \( \forall  n  \) функция \( P_n(x ) \) является многочленом степени \( n \) с положительным старшим коэффициентом.
\end{lemma}

\begin{proof} \(  \) 

    По индукции: 

    База: \( \begin{aligned}
    n = 0 ,\quad  P_0 =1 \\ 
    n=1 ,\quad  P_1 =  x
    \end{aligned} \) 

    Шаг: для \( P_n(x ) \) верно, докажем для \( P_{n+1} (x) \): 

    \[ P_{n+1 } (x ) = \frac{(2 n +1 )}{n+1      } \underbrace{x P_n(x )}_{n+ 1 \text{ степень} } - \frac{n }{n+1 } \underbrace{P_{n-1 }  (x )}_{n-1 \text{ степень} }   \] 
    \[ P _{n+1 }  ( x) \text{  - многочлен степени }  n+ 1  \] 

    \[ \frac{2n -1 }{n+1 }x P_n \text{  - имеем положительную старую степень}   \] 
\end{proof}

Дифференцируем  \( w(x,t ) \) по \( x \): 

\[ \frac{\partial  w }{\partial  x } =  -\frac{1}{2}  \frac{- 2t  }{(1 - 2xt + t ^2 )^{\frac{3}{2} } }  \] 
\[ (1 - 2 xt +t ^2  ) \frac{\partial  w}{\partial  x } - t w = 0   \] 
\[ (1 - 2 xt  +t ^2  ) \sum_{n =0 }^{ \infty  } P_n ' (x ) t ^{ n }  - t \sum_{n =0 }^{\infty  } P_n(x ) t^n = 0  \] 

\[ (A) : \text{ }  P_n '(x ) -2 x P_{n+1 } ' (x ) + P_{n-2 } ' (x ) - P_{n-1 } (x ) =0 ,\text{ } \forall  n \ge 2 \quad \left( \frac{d }{dx }(*)  \right)  \] 

\[ n \to  n+1 : \text{ }P_{n +1 } '(x ) -2 x P_{n+2  } ' (x ) + P_{n-1 } ' (x ) - P_{n} (x ) =0   \] 

\[ (B) : (n+1 ) P_{n+1 } ' (x ) - (2 n - 1 ) P_n (x ) - (2n + 1 ) xP_n '(x ) + n P_{n-1 }  ' (x ) = 0 \] 

\[ (B ) - (n +1 )(A ) : \text{ }   [- (2n +1 )+ (n+1 )] P_n(x ) - x [(2 n +1 )-2 ( n+1 )]P_n '(x ) + [n - (n+1 )]P_{n-1 } '(x) = 0\] 

\[ (V): \text{ } - n P_n(x ) + x P_n '( x ) - P_{n-1 }  '(x ) = 0 \] 

\[ (B ) - n (A ) : \text{ }  P_{n +1 }  ' ( x ) - [ ( 2 n+ 1 ) - n ] P_n ' ( x ) - x [(2 n +1 ) - 2n ] P_n ' (x ) = 0 \] 

\[ (G): \text{ }  P_{n+1 }  '( x )  -(n +1 ) P_n (x ) - x P_n ' (x ) =0 \] 

\[ (V )+ (G ) : \text{ }  P_{n+1 } '( x ) - (2 n +1 ) P_n (x ) - P_{n -1 }  ' ( x ) =0 \] 

\[ (2 n+ 1 )    P_n ( x ) = P_{n+1  }  '(x ) - P_{n-1 }  ' (x ) \] 

\section{Дифференциальное уравнения. Соотношения ортогональностей }

\[  - (V )  :\text{ } n P_n(x ) - x P_n ' (x ) + P_{n-1 } '(x )  \] 

\[ (G ) \text{ }  n+1 \to  n : \text{ }  P_n '(x ) - n P_{n -1 } (x ) - x P_{n-1 }  '(x ) =0  \] 

Суммируем наши фигни: 

\[ xn P_n(x ) - x P_n ' + x P_{n-1 } ' (x ) + P_{n }  ' (x ) - n P_{n-1 }  (x) - x P_{n-1 } ' (x )=0\] 
\[ (1- x ^2 ) P_n' (x ) + n x P_n (x )- n P_{n-1 }  (x ) =0 \bigg| \cdot \frac{d}{dx} \] 

\[ [(1 + x ^2 )P_n ' ]' + n P_n(x ) + n x P_n '(x ) \underbrace{- n P_{n-1 } ' (x )}_{n ^2 P_n - nx P_n} =0  \] 

То есть многочлен Лежандра является частным решением линейного дифференциального уравнения второго порядка: 

\[ [(1 - x ^2 y ' ) ] ' + n (n+1 ) y = 0 \] 

\[ L_2 ^ h (-1 ,1 ) , \text{ }  h=1 : \text{ }  (f, g ) = \int_{-1 }^{1 }  fg dx  \]  

\[ ((1 -x ^2 )P_n ') ' + n (n+1 ) P_n =0  \text{ } | \cdot P_m  \] 
\[ ((1- x ^2 ) P_m ' ) ' m (m +1 ) P_m = 0 \text{ } | \cdot P_n  \] 

\[ \underbrace{[(1 -x ^2 ) (P_m P_n ' - P_n P_m ' )]'}_{(1)} + (n (n +1 ) - m(m+1 )) P_m P_n =0 \] 

\[  \int_{-1 }^{1 }  (1) dx \to 0 , \quad  \int_{-1 }^{1 }  P_n P_m = 0 , \text{ при } n \neq m    \] 

, ортогональность доказана.

\[ \left\lVert P_n   \right\rVert ^2 = (P_n P_m ) = \int_{-1 }^{1 }  P_n  ^2(x) dx  \] 

Замена в \( (*) \text{ }  n+ 1 \to  n \): 

\[ (\tilde{*}): \text{ } n P_n - (2 n -1 ) x P_{n-1 }  +(n-1 ) P_{n-2 }  =0\] 

\( (* ) (2n-1 ) P_{n-1 } +  (\tilde{ * } ) (2n +1 ) P_n: \) 

\[ (2n -1 )(n+1 ) P_{n+1 }  P_{n-1 }  + (2n -1 ) n P_{n-1  }  ^2 - (2n +1 ) (n+1 ) P_{n-2 } P_n =0  \] 

\[ (2n -1 ) \int_{-1 }^{1 }  P_{n-1 }  ^2 ( x ) dx = (2 n +1 ) \int_{-1 }^{1 }  P_n  ^2 (x ) dx , \text{ }  \forall  n \ge 2 \]  
\[ \int_{-1 }^{1 }  P_n ^2 dx = \frac{ 2n -1 }{2n +1 } \int_{-1 }^{1 } P_{n-1  }  ^2 dx = \frac{2n -1 }{2n +1 } \frac{ 2n -3 }{2n -1 } \int_{-1 }^{1 }  P_{n-2 }  ^2 dx = ...    \] 
\[ ... =\frac{\cancel{2n -1} }{2 n+1 } \frac{ 2n-3 }{\cancel{2n -1} } \ldots \frac{3}{5 }  \underbrace{\int_{-1 }^{1 }  P_n ^2 (x ) d x}_{\frac{2}{3} } = \frac{2}{2n +1 }     \] 

\[ \left\lVert P_n  \right\rVert ^2 = \frac{2}{2n +1 }  , \text{ }  n \ge 2  \] 

\[ \int_{-1 }^{1 }  P_n P_m dx   = \frac{2}{2n +1 }  \delta_{nm}  \] 

\section{Формула Родрига и теорема о разложении функций в ряд по  многочленам Лежандра}

\begin{theorem}[Формула Родрига]
    \( \forall       n \ge 0 : \displaystyle  P_n(x ) = \frac{1}{2^n n! }  \frac{d ^n }{d x ^n } ( x ^2 -1 ) ^ n  \) 
\end{theorem}

\begin{proof}
    Руслан не буянь здесь доказательство не нужно, у нас в программе это не требуется.
\end{proof}

\begin{theorem}[Теорема о разложении функции в ряд по многочленам Лежандра]
    Пусть \( f: [ -1, 1 ] \to  \mathbb{R}    \)  непрерывно дифференцируемая функция. Тогда \( \forall  x \in  [-1 ,1 ]  \)  справедливо равенство: 

    \[ f(x ) = \sum_{n= 0 }^{\infty  } c_n P_n (x)  \] 
    , где \( P_n(x) \) - многочлен Лежандра стандартизированный  с помощью производящей функции \( w(x,t) \) 

    \[ c_n = \frac{(f, P_n )}{(P_n,P_n)}  = \left( n+\frac{1}{2 }  \right) \int_{-1 }^{1 }  f(x ) P_n (x ) dx  \] 
\end{theorem}

%%-------------------------------%%

% Закрытие документа, если файл компилируется отдельно
\ifdefined\mainfile
    % Если это основной файл, не нужно заканчивать документ
\else
    \end{document}
\fi