% Условная компиляция для самостоятельной работы
\ifdefined\mainfile
    % Если это часть основного файла, не добавляем начало и конец документа
\else
    \documentclass[12pt, a4paper]{report}
    \usepackage{/Users/vladbelousov/Desktop/Semestr_4-FP-NSU/Настройка/library}
    \usepackage[utf8]{inputenc} % Подключение поддержки UTF-8
    \begin{document}
\fi

%%-------------------------------%%

- Трехчленная рекуррентная формула 

Пусть \( q_n( x ) = a_n x^n + b_n x^{n -1} + ...    \) Тогда справедливо представление: 

\[ x q_n(x ) = \frac{a_n }{a_{n+1 }} q_{n+1 }  (x ) + \left( \frac{b_n }{a_n } - \frac{b_{n+1 } }{a_{n +1 } }   \right)q_n(x ) + \frac{a_{n -1 } }{a_n }q_n(x )   \] 

\begin{proof}
    \[  \] 
    Разложим многочлен степени \( n+1  \) по ортогональным многочленам: 

    \[ x q_n (x ) = \sum_{m =0 }^{n +1 } c_{n m } q_m (x ) \] 
    
    откуда \( c_{nm } = 0  \) при \( m>n +1  \) при этом 
    
    \[ c_{ nm }  = (x q_n , q_m ) = \int_{a }^{b } x q_n (x ) q_m (x ) h(x )dx = (x q_m , q_n ) = c_{mn }    \] 
    
    откуда \( c_{nm } = 0  \) при \( m < n-1  \). Получаем 
    
    \[ x q_n (x ) = c_{c ( n+1 )}   q_{ n+1 }  (x ) + c_{nn }  q_n (x ) + c_{n (n -1 )}     q_{n - 1 }  (x )  \] 
    
    остается вычислить коэффициенты. Подставим в предыдущую формулу: 
    
    \[ q_n (x  ) = a_n x^n   + b_n x^{ n -1 } +... \] 
    
    Получим: 
    
    \[ x(a_n x^n + b_n x^n + ... )= c_{n (n +1 )}  (a_{n+1 } x^{n+1 } + b_{n+1 } x^{n +1 } +...      )+ c_{nn } (a_n x^n + b_n x^{n-1}  + ... ) + \] 
    \[ +c_{n (n-1 )} (a_{n-1 } x^{n-1 } + b_{n-1} x^{n-2 } +... ) \] 
    
    Собираем коэффициенты при одинаковых степенях: 
    
    \[ a_n = c_{n (n +1 )}  a_{n+1 } (\text{при } x^{n+1 }) \Rightarrow c_{n (n+1 )}    = \frac{a_n }{a_{n+1} }   \]  
    
    \[ b_n = c_{n (n +1 )} b_{n +1 } c_{nn   } a_n (\text{при } x^n ) \Rightarrow c_{nn } = \frac{b_n -\frac{a_n }{a_{n+1}  }b_{n+1}     }{a_n}    \] 
    \[ ... \] 
    
    По симметрии находим \( \displaystyle  c_{n(n +1)}   = c_{n (n-1 )}     = \frac{a_{n- 1 } }{a_n} \) 

\end{proof}

Огрубляя ситуацию, можно сказать, что для любой последовательности ортогональных многочленов \( q_0 ,q_1 ,q_2, \ldots, q_n, \ldots,  \) существует постоянные \( A_n , B_n ,C_n  \) такие, что: 

\[ q_{n+1 }(x )= (A_n x + B_n ) q_n (x ) + C_nq_{n-1 } ( x) \] 

\begin{proposition}
    Все ортогональные многочлены степени \( n  \) имеют ровно \( n  \)  корней, причем эти корни (нули многочлена \( q_n \)) действительны, просты и расположены внутри интервала \( (a, b) .\) 
\end{proposition}

\begin{proof}
    \[  \] 
    
    Предположим противное: существует только \( k<n  \) точек, в которых \( q_n  \) меняет знак. При этом как минимум одна смена знака есть в силу: 

    \[ \int_{a }^{b } q_0 (x ) q_n (x ) h(x )dx = 0 ,\quad  \forall  n \ge 1 \] 

    при этом \( q_0      \) - это константа, а \( h(x ) \ge 0  \), значит, многочлен \( q_n \) принимает на \( (a,b) \) значения разных знаков. Обозначим нули \( q_n \)  как \( x_1 , x_2, \ldots, x_k \) 

    Введем многочлен \( P_k (x )= (x - x_1 )...(x- x_k) \), тогда многочлен \( q_n P_k(x ) \) сохраняет знак и значит: 

    \[ \int_{a }^{b } q_n(x ) P_k(x ) h(x )dx \neq 0  \] 

    что противоречит свойству ортогональности многочлена \( q_n \) любому многочлену степени, меньшей \( n \) (Если \( P_m(x) \) - произвольный многочлен степени \( m  \), и \( n>m \), то \( q_n \perp P_m \))
\end{proof}

\begin{flushleft}  
    \textbf{Следствие 1.} из утверждения и рекуррентной формулы: 

    - Два соседних многочлена не имеют общих корней. 

    Предположим противное: \( q_n (x_0)  = q_{n +1 } (x_0 ) = 0\). Воспользуемся рекуррентной формулой: 

    \[ x q_n (x ) = \frac{a_n } {a_{n+1 } } q_{n+1 } (x ) + \left(  \frac{b_n }{a_n } - \frac{b_{n +1 } }{a_{n +1 } }   \right) q_{n } (x_0 ) + \frac{a_{n - 1 } }{x_n }  q_{n -1 } (x_0) \] 

    то есть 

    \[ 0 = \frac{a_{n -1 } }{a_n } q_{n -1 } (x_0 )  \] 

    Значит, \( x_0  \)  - корень \( q_{n-1}  \). Рассуждая аналогично, \( x_0  \) - корень \( q_{n-2 } ,..., q_0  \), что противоречит свойству многочлена \( q_0 \), равного константе \(\displaystyle  \int_{a }^{b } q_0 ^2 (x )h(x )dx =1  \) 
\end{flushleft}

\begin{flushleft}  
    \textbf{Следствие 2.} 

    - Если \( x_0  \)  - корень многочлена \( q_n    \), то соседние многочлены \( q_{n -1}  \) и \( q_{n+1}  \) принимают в точке \( x_0 \) значения разных знаков. 

    Пусть \( q_n(x_0 ) = 0 \). Воспользуемся рекуррентной формулой 

    \[ x_0 = \frac{a_n }{a_{n+1 } } q_{n+1 } (x_0 ) + \left(  \frac{b_n }{a_n } - \frac{b_{n+1 } }{a_{n+1 } }   \right) 0 + \frac{a_{n -1 } }{b_{n-1 } } q_{n-1 }   (x_0)   \] 

    то есть 

    \[ \frac{a_n }{a_{n +1 } } q_{n +1 } (x_0 ) = - \frac{ a_{n -1 } }{a_n }q_{n+1 } (x ) = - \frac{ a_{n -1 } }{a_n }q_{n-1 } (x_0)    \]  

    причем \( a_m    \) - старший коэффициент полинома \( q_m \), положительный по построению.  
 
\end{flushleft}

\begin{flushleft}  
    \textbf{Следствие 3.}

    -Корни многочлена \( q_n \) лежат между корнями многочлена \( q_{n+1}  \) 
\end{flushleft}

\section{Классические ортогональные многочлены }

Наши основные многочлены:  

\[
\begin{array}{|c|c|c|c|}
\hline
\text{Название} & \text{Обозначение} & \text{Интервал ортогональности} & \text{Весовая функция} \\
\hline
\text{Эрмитовы} & H_n(x) & \mathbb{R} & e^{-x^2} \\
\hline
\text{Лагерра} & L_n^{\alpha}(x) & (0, +\infty) & e^{-x} \\
\hline
\text{Лежандра} & P_n(x) & (-1, 1) & 1 \\
\hline
\end{array}
\]


\begin{definition}
    Функцию \( w( x, t ) \)  двух переменных называют производящей функцией для последовательности многочленов \( q_0 , q_1, \ldots, q_n, \ldots,  \) если ее разложение в ряд по степеням \( t \)  при достаточно малых \( t \) имеет вид: 

    \[ w(x,t )= \sum_{n =0 }^{\infty } \frac{q_n(x )}{a_n } t^n    \]  

    где \( a_n  \) - некоторые постоянные.
\end{definition}

Под "классическим" ортогональными многочленами мы понимаем только то многочлены, весовая функция которых удовлетворяет уравнению Пирсона: 

\[ \frac{h' (x )}{h(x )} = \frac{ \alpha_0 \alpha_1 x }{\beta_0 + \beta_1 x + \beta_2 x ^2 }   \] 

и предельным условиям 

\[ \lim_{x  \to a+\infty } h(x ) B(x ) = \lim_{x \to b -0 } h(x )A(x ) = 0  \] 

где \( B(x ) = \beta_0 + \beta_1 x + \beta_2  x ^2 , \text{ }  A (x ) = \alpha_0    + \alpha_1 x \) 


Если весовая функция h которых удовлетворяют уравнению Пирсона и граничныи условиям, то 

- ортогональный многочлен \( q_n      \) является решением дифференциального уравнения 

\[ B(x ) y''(x )+ [A(x )+ B' (x )]y'(x )- \gamma_n y(x )= 0\] 

где \( \gamma_n = n [ \alpha_1 + (n+1 )\beta_2 ] \) 

- имеет место формула Родрига: 

\[ q_n(x ) = c_n \frac{1 }{h(x )} \frac{d^n }{d x ^n } [ h(x )B^n (x )] , \quad  n=0,1,2,\ldots,  \] 

где \( c_n \) - некоторые постоянные. 

- производные \( \displaystyle  \frac{ d^m }{d x^{m} }  [q_n(x )]\) являются классическими ортогональными многочленами с тем же промежутком ортогональности 

- у многочленов \( q_0,q_1 ,q_2, \ldots, q_n, \ldots,  \) существует производящая функция, выражающаяся через элементарные функции.

Способы задания ортогональных многочленов: 

- ортогонализация  мономов в \( L_2^{h } (a,b ) \) 

- решение дифференциального уравнения для соответствующего \( n \) 

- формула Родрига 

- рекуррентное соотношение (нужно знать \( q_0,q_1 \))

- разложение производящей функции.


%%-------------------------------%%

% Закрытие документа, если файл компилируется отдельно
\ifdefined\mainfile
    % Если это основной файл, не нужно заканчивать документ
\else
    \end{document}
\fi