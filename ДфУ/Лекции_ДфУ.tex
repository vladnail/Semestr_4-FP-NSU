\documentclass[12pt, a4paper,oneside]{book}
% Подключение библиотеки
\usepackage{/Users/vladbelousov/Desktop/Semestr_4-FP-NSU/Настройка/library}

\makeatletter
\@addtoreset{corollarycounter}{section}
\@addtoreset{theorem}{section}
\@addtoreset{definition}{section}
\@addtoreset{lemma}{section}
\@addtoreset{proposition}{section}
\makeatother  

\begin{document}

\begin{titlepage}
    \thispagestyle{empty}  % Отключаем нумерацию страницы на титульном листе
    \centering
    \vspace*{1cm}  % Отступ сверху

    \textbf{\huge Конспект лекций по дисциплине}  \\[1.5cm]  % Название
    \textbf{\Huge Дифференциальные уравнения}  \\[2cm]   % Название дисциплины (оставьте пустым для добавления вручную)
    \textbf{\Large Новосибирский государственный университет} \\[0.5cm]
    \textbf{\large Физический факультет} \\[0.5cm]
    \textbf{\large 4-й семестр} \\[0.5cm]
    \textbf{\large 2025 год} \\[10cm]

    \begin{flushright}
        \large Студент: Б.В.О \\[0.5cm]  % Ваше имя
        Преподаватель: Скворцова Мария Александровна  % Ф.И.О. преподавателя
    \end{flushright}
\end{titlepage}

\tableofcontents  % Создание оглавления

\def\mainfile{}  % Определяем макрос для основного файла
%Февраль
\input{Лекции_по_дням/04.02_ДфУ.tex}
% Условная компиляция для самостоятельной работы
\ifdefined\mainfile
    % Если это часть основного файла, не добавляем начало и конец документа
\else
    \documentclass[12pt, a4paper]{report}
    \usepackage{/Users/vladbelousov/Desktop/Semestr_4-FP-NSU/Настройка/library}
    \usepackage[utf8]{inputenc} % Подключение поддержки UTF-8
    \begin{document}
\fi

%%-------------------------------%%

\[ \begin{aligned}
\begin{cases}
    \displaystyle I[y] = \int_{x_0}^{x_1} F(x,y(x),y'(x))dx \\
    y(x_0)=y_0, \text{ } y(x_1)=y_1
\end{cases}
\quad (1)
\end{aligned} \] 

Найти функцию \( y(x) \)   такую, чтобы функционал \( I[y] \) принимал наибольшее или наименьшее значение. 

Необходимо найти условие локального экстремума: 

Если \( \tilde{y }       \)  экстремум \( \Rightarrow \tilde{y }  \frac{\partial F }{\partial y} - \frac{d}{dx }  \frac{\partial F}{\partial y '} = 0 \quad (2)   \)

\begin{proof}[Докозательство формулы (2)]
    \[ I[\tilde{y}] \le I[y] , y = \tilde{y} + \varepsilon \eta , \varepsilon \in  (- \varepsilon_1, \varepsilon_1), \eta(x) \in C^2([x_0,x_1])  - \text{финитная }    \] 
    % рисунок 
    \[ \underbrace{I[y]}_{= g(0)} \le \underbrace{I[\tilde{y}+ \varepsilon \eta ]}_{=g(\varepsilon)} \Rightarrow g(0) \le g(\varepsilon) \Rightarrow g'( 0) = 0\] 

    \[ 0= \frac{d}{d \varepsilon} g(\varepsilon) |_{\varepsilon= 0 }= \int_{x_0}^{x_1} \left( \frac{\partial F }{\partial y } ( x , \tilde{y}(x) , \tilde{y }' (x) ) - \frac{d}{dx }  \frac{\partial F }{\partial y' } ( x , \tilde{y}(x) , \tilde{y }' (x) )  \right) \eta (x) dx , \forall \eta (x)  - \text{финитная}    \] 

    \begin{lemma}[Лагранжа]     
        Пусть \( f(x) \) - непрерывна и \( \displaystyle \int_{x_0}^{x_1} f(x) \eta (x) dx =0. \forall \eta (x) \) - финитная на \( [x_0,x_1] \). Тогда \( f(x) = 0 , \forall x \in  [x_0,x_1] \) 
    \end{lemma}

    \begin{proof}
        От противного:
        %рисунок 
        Пусть для определенности \( f(\tilde{x } )> 0        \) . Тогда так как \( f(\tilde{x}) \) - непрерывна, то \( f(x)>0    \) при \( x \in (\tilde{x}- \delta_0, \tilde{x}+\delta_0) \) 

        Возьмем функцию \( \eta(x) = \begin{aligned}
            \begin{cases}
                (\delta_0 ^2 - (x- \tilde{x})) ^ 4 , |x- \tilde{x}|< \delta   \\
                0 , |x- \tilde{ x } | > \delta_0
                \end{cases}
                - \text{финитная функция} 
        \end{aligned}\)
        %рисунок

        \[ \int_{ x_0 }^{x_1} f(x ) \eta(x ) dx = \int_{x_0 - \delta_0}^{x_1- \delta_0} \underbrace{f(x)}_{>0} \underbrace{\eta( x)}_{>0} dx >0 - \text{противоречие}    \] 

        \[ \Rightarrow \forall x \in  [x_0, x_1] : f(x) = 0  \] 
    \end{proof}

    Из доказательства леммы следует, что доказана формула (2)
\end{proof}

\section{Случай понижения порядка в уравнении Эйлера}   

\[ \frac{\partial F}{\partial y} - \frac{d}{d x }  \frac{\partial F}{\partial y'} = 0 \quad  F=F(x,y(x),y'(x))  \] 

1) \(\displaystyle  F= f(x,y) \Rightarrow \frac{\partial F }{\partial y} (x,y) =0 \Rightarrow y= y(x)    \) 

2) \( \displaystyle F= F (x, y ') \Rightarrow \frac{d}{dx }  \frac{\partial F }{\partial y'} (x,y') =0 \Rightarrow \frac{\partial F }{\partial y'} (x,y') = C    \) 


3) \( F= F( y , y ') \) :
%%-------------------------------%%

% Закрытие документа, если файл компилируется отдельно
\ifdefined\mainfile
    % Если это основной файл, не нужно заканчивать документ
\else
    \end{document}
\fi
% Условная компиляция для самостоятельной работы
\ifdefined\mainfile
    % Если это часть основного файла, не добавляем начало и конец документа
\else
    \documentclass[12pt, a4paper]{report}
    \usepackage{/Users/vladbelousov/Desktop/Semestr_4-FP-NSU/Настройка/library}
    \usepackage[utf8]{inputenc} % Подключение поддержки UTF-8
    \begin{document}
\fi

%%-------------------------------%%

\section{ Вариационная задача с высшими производными }

\[ I[y ]= \int_{x_0 }^{x_1 } F( x, y (x), y' (x),..., y ^{(n )} (x))  dx\]  

\[ \begin{cases}
    y(x_0) = y_0 ,\quad  y (x_1) = y_1 \\
    y ' (x_0) = y ' _0 , \quad y ' (x_1) = y ' _1 \\
    \ldots 
    y^{(n )} = y^{(n )}_0, \quad y^{(n )} (x_1) = y^{(n )}_1
\end{cases} \]  

Необходимое условие локального экстремума: 

Если функция \( \tilde{y }(x ) \)  доставляет функционалу локальному экстремум, то \( \tilde{ y }(x) \)  - решение диффернциального уравнения

\[ \frac{\partial  F }{\partial  y } - \frac{d}{dx } \frac{ \partial  F }{\partial  y ' } + \frac{d ^2 }{dx ^2 } \frac{\partial  F }{\partial  y ''} + \ldots + (-1 )^{n}  \frac{d ^n }{dx ^n } \frac{\partial  F }{\partial  y ^{(n )}} = 0    \] 


\begin{proof}
    \[   \] 
    Пусть \( \tilde{y }( x) \)  доставляет функционалу локальный минимум \( \Rightarrow  \exists \varepsilon_0 >0 , \text{ }  \forall  y(x )\), удовлетворяет краевым условиям, \( \displaystyle \sup_{x \in [x_0 , x_1 ] } |y(x)-\tilde{y }(x) | < \varepsilon_0  \Rightarrow I[\tilde{y }] \le  I[y] \) 

    Возьмем \( y(x ) = \tilde{y }( x ) + \varepsilon \eta ( x ), \eta(x) \)  финитная функция.

    \[ \underbrace{I[\tilde{ y }]}_{g(0)} \le  \underbrace{I[\tilde{y }+ \varepsilon \eta ]}_{g(\varepsilon)} \Rightarrow g(0 ) \le  g(\varepsilon) \Rightarrow \varepsilon = 0  - \text{ точка локального минимума для функции } g(\varepsilon) \] 

    \[ g ' (0 ) = 0 \]  

    \[ 0 = \frac{d}{d \varepsilon } g(\varepsilon ) \bigg |_{\varepsilon = 0 }  = \frac{d}{d \varepsilon} \int_{x_0 }^{x_1} F ( x, \tilde{y }( x)  + \varepsilon \eta ( x ), \tilde{y }' ( x) + \varepsilon \eta' ( x ), ...) dx \bigg |_{\varepsilon = 0} \] 

    Если \( F \in C^{\infty  } (\mathbb{R} ^{n +2 } ), y(x ) \in C^{\infty  } ([x_0 , x_1 ])    \), то 

    \[ 0 = \int_{x_0 }^{x_1} \left[ \frac{\partial  F }{\partial  y } \eta( x) + \frac{\partial F }{\partial  y' }(... )\eta' ( x )+ \frac{\partial F }{\partial y ''} (...) \eta''(x) +\dots  \right]  dx \bigg | _{\varepsilon = 0 }  =    \] 

    \[ =\int_{x_0 }^{x_1}  \frac{\partial  F }{\partial  y } \eta ( x ) dx + \frac{\partial  F } {\partial  y ' } \eta  ( x ) \bigg |_{x_0 }^{x_1}   - \int_{x_0 }^{x_1} \eta( x ) \frac{d }{dx } \frac{\partial  F }{\partial  y ' } (... )dx + \frac{\partial  F }{ \partial  y ' } (... ) \eta ( x ) \bigg |_{x_0 }^{x_1} - \int_{x_0 }^{x_1}  \eta ' ( x ) \frac{d}{dx } \frac{\partial  F }{\partial  y ''} dx ... \text{ и тд}    \] 

    \[ \text{Для n =2:}  \int_{x_0 }^{x_1} \left(  \frac{\partial  F } {\partial  y } - \frac{d}{dx } \frac{\partial F }{\partial  y '}  \right) \eta(x )dx - \eta(x ) \frac{d}{dx } \frac{\partial F }{ \partial  y '}\bigg |_{x_0 }^{x_1 } \int _{x_0 }^{x_1} \eta  ( x ) \frac{d ^2 }{dx ^2 } \frac{\partial F }{\partial y ''} dx       =  \] 

    \[ =\int_{x_0 }^{x_1} \left(  \frac{\partial  F } {\partial  y } - \frac{d}{dx } \frac{\partial F }{\partial  y '} + \frac{d ^2 }{d x ^2 } \frac{\partial F }{\partial y ''}  \right) \eta(x )dx   = 0 \quad \forall \text{ финитной функции  } \eta( x) \] 

    Если n = 2 , то по лемме Лагранжа: \( \displaystyle  \frac{\partial  F } {\partial  y } - \frac{d}{dx }  \frac{\partial  F } { \partial y ' } + \frac{d ^2 }{dx ^2 }  \frac{\partial  F }{ \partial  y''}  =0   \) 

    При n > 2 аналогично.

\end{proof}


\section{Вариационная задача с несколькими независимыми переменными}

\[ \begin{cases}
    \displaystyle I [ z] = \iint_{D } F (x, y , z(x ), z_x ' ( x,y), z_y ' ( x,y )) dx dy \\
    \displaystyle z  | _{(x,y ) \in  \partial D } = \varphi ( x,y)
\end{cases} \] 

Необходимое условие локального экстремума: 

\[ \frac{\partial  F }{ \partial z } - \frac{\partial  }{\partial  x } \frac{\partial  F }{ \partial z_x ' } - \frac{\partial  }{\partial  y } \frac{\partial  F }{ \partial z_y ' } = 0    - \text{ уравнение Эйлера-Остроградского}   \] 

\textbf{Без доказательства} 

\section{Принцип Остроградского-Гамильтона (принцип наименьшего действия, признак стационарного действия, основной вариационный принцип механики)}

T - кинетическая энергия, U - потенциальная энергия: 

\[ L = T - U - \text{ функция Лагранжа (Лагранжиан)}  \] 

\[ S = \int_{t_0 }^{t_1} L dt - \text{функционал действия}  \] 

Движения в системе происходит  по экстремалям функционала действия. 

Пример: 

\begin{center}
    \includegraphics[width=0.3\textwidth]{/Users/vladbelousov/Desktop/Semestr_4-FP-NSU/ДфУ/Лекции_по_дням/image/17.png}
\end{center}

\[ T = \frac{m \dot{x } ^2 }{2} \quad  U = \frac{k x ^2 }{2}  \] 

\[ S = \int_{t_0 }^{t_1} \left( \frac{m \dot{x } ^2 }{2} -  \frac{k x ^2 }{2}\right)  dt \] 

Уравнение Эйлера (уравнение Лагранжа): \( \displaystyle  \frac{\partial L }{ \partial x  } - \frac{d}{dt} \frac{\partial L }{ \partial \dot{x}  } = 0  \) 

\[ - kx - \frac{d}{dt } ( m \dot{x } ) = 0  \Rightarrow m \ddot{x} + kx = 0  \] 

Понижение порядка: \( \displaystyle  L - \dot{x } \frac{\partial L}{\partial  \dot{x} } = C  \) 

\[ \frac{m \dot{x }  ^2 }{2 } - \frac{k x ^2 }{2 } - \dot{x } m \dot{x } = c \Rightarrow -\frac{m \dot{ x } ^2 }{2 } - \frac{k x ^2  }{2}  =c  - \text{ З.С.Э.}   \] 

\section{Изопериметрическая задача }

Найти кривую заданной длины, ограничивающую наибольшую площадь.

\[ \begin{aligned}
    S &\to  \max  \\ 
    l &= \mathrm{const}  
\end{aligned} \] 

\begin{center}
    \includegraphics[width=0.5\textwidth]{/Users/vladbelousov/Desktop/Semestr_4-FP-NSU/ДфУ/Лекции_по_дням/image/18.png}
\end{center}

\[\begin{aligned}
    \begin{cases}
        x = x(t) \\ 
        y = y(t) \quad  t \in [ t_0 ,t_1]
        \end{cases}
    \begin{aligned}
        x(t_0 )=x(t_1) \\ 
        y(t_0 )=y(t_1) \\  
    \end{aligned}
\end{aligned} \]

\[ S = \iint_D dx dy   \] 

\[\text{ Формула Грина: }  \int_{ \partial D} \left( P(x,y )dx +Q(x,y )dy  \right) =\iint _D \left( - \frac{\partial P }{ \partial  y } (x,y ) + \frac{\partial Q}{\partial x }(x,y)   \right) dx dy\] 

\[ S = \iint_D dx dy = \iint_D \left(\underbrace{ \frac{1}{2 }}_{- \frac{\partial  P}{\partial y} } + \underbrace{ \frac{1}{2 }}_{ \frac{\partial  Q}{\partial x} }       \right)dx dy = \int_{ \partial  D } \left(  - \frac{y}{2 } dx + \frac{x}{2 } dy \right)  =\frac{1}{2 } \int_{t_0 }^{t_1} (x(t )y ' (t )- x' (t )y (t))dt  \] 

\[ \begin{cases}
\displaystyle \frac{\partial P } {\partial y } = -\frac{1}{2 }  \quad   P = - \frac{y}{2}  \\
\displaystyle \frac{\partial Q }{\partial x } = \frac{1}{2 } \quad  Q = \frac{x}{2}   
\end{cases} \] 

\[ l = \int_{t_0 }^{t_1} \sqrt{(x' (t ) ^2 + (y '(t ) )^2 } dt = \mathrm{const}   \] 

Задача из математического анализа : 

\[ \begin{cases}
f(x_1, \ldots, x_n ) \to  \mathrm{extz}   \quad \quad  \tilde{f } =f + \lambda_1 g_1 + ... + \lambda_m g_m \to  \mathrm{extz}   \\
g_1 (x_1, \ldots, x_n ) = 0 \\
\vdots \\
g_m (x_1, \ldots, x_n ) = 0
\end{cases} \] 

Задача вариационного исчисления: 

\[ I[y_1, \ldots, yn] = \int_{x_0 }^{x_1 } F(x,y_1, \ldots, y_n,y_1',..., y_n') dx \to  \mathrm{extz}   \] 

\[ \begin{cases}
y_1(x_0 ) = y_0 ^1 \quad  y_n(x_0 ) = y_0 ^n \\
y_1(x_1 ) = y_1 ^1 \quad  y_n(x_1 ) = y_1 ^n \\
\end{cases} \] 

\[ Y[y_1, \ldots, y_n ] = \int_{x_0 }^{x_1} G(x,y_1, \ldots, y_n,y_1',..., y_n')dx = \mathrm{const}    \] 

\textbf{Необходимое условие локального экстремума:} 

Пусть \( \tilde{y_1 }(x ),..., \tilde{y_n}(x) \) доставляет локальный экстремум функционалу \( I[y_1, \ldots, y_n] \)  и не является экстремалью функционалу \( Y[y_1, \ldots, y_n] \), тогда \( \exists  \lambda \in  \mathbb{R} \), такие, что \( \tilde{y_1}(x),..., \tilde{y_n}(x) \) доставляют экстремум функционалу \( \tilde{I } = I = \lambda Y \) 

\textbf{Без доказательства}

Замечание. \( I + \lambda Y \to  \mathrm{extz}  \Leftarrow \begin{cases}
    Y   = \mathrm{const}  \\
    I\to  \mathrm{extz} 
    \end{cases}  \) 

\[ \lambda \left( \frac{1}{\lambda }I + Y   \right) \to  \mathrm{extz} \Leftrightarrow Y+ \frac{1}{\lambda } \to  \mathrm{extz}  \Leftarrow \begin{cases}
Y \to  \mathrm{extz}  \\
I = \mathrm{const}  
\end{cases} \] 

Двойственная  задача: 

\[ \begin{aligned}
\begin{cases}
S \to  \max  \\
l = \mathrm{const}
\end{cases}
\Leftrightarrow 
\begin{cases}
l \to  \min  \\
S = \mathrm{const}
\end{cases}
\end{aligned} \]

\section{Решение классической изопериметрической задачи}

\[ \tilde{I } = S + \lambda l = \int_{t_0 }^{t_1 } \underbrace{\left[ \frac{1}{2 } (x y ' - x' y )+ \lambda \sqrt{(x') ^2 + (y ' ) ^2 } \right]}_{F}dt \to  \mathrm{extz}  \] 

\[ \begin{aligned}
    \begin{cases}
        \displaystyle \frac{\partial F }{\partial x } - \frac{d}{dt } \frac{\partial  F }{\partial  x ' } = 0 \quad  \\ 
        \vspace{0.01 mm} \\
        \displaystyle \frac{\partial F }{\partial y } - \frac{d}{dt } \frac{\partial  F }{\partial  y ' } = 0 \quad 
    \end{cases}
    \begin{cases}
        \displaystyle  \frac{1}{2 } y ' - \frac{d}{dt } \left[ -\frac{1}{2 } y + \lambda \frac{x ' }{\sqrt{(x') ^2 + (y ' ) ^2 }}  \right] =0\\
        \displaystyle -\frac{1}{2 } x ' - \frac{d}{dt } \left[ \frac{1}{2 } x + \lambda \frac{x ' }{\sqrt{(y') ^2 + (y ' ) ^2 }}  \right] =0 
    \end{cases}
\end{aligned}
\] 

№ 39 (задачник Александрова-Егорова).  Понизить порядок не получится так же, как в простейшей задаче.

\[ \begin{aligned}
    \begin{cases}
        \displaystyle \frac{d}{dt } \left[ \frac{y}{2 } + \frac{y}{2 } - \lambda \frac{x' }{\sqrt{(y') ^2 + (y ' ) ^2 }}  \right] =0 \\
        \displaystyle -\frac{d}{dt } \left[ \frac{x}{2 } + \frac{x}{2 } + \lambda \frac{y' }{\sqrt{(x') ^2 + (y ' ) ^2 }}  \right] =0
    \end{cases} 
    \begin{cases}
    \displaystyle y - c_1 = \frac{\lambda x' }{\sqrt{(x') ^2 + (y ' ) ^2 }} \\
    \displaystyle x - c_2 = \frac{-\lambda y' }{\sqrt{(x') ^2 + (y ' ) ^2 }}
    \end{cases}
\end{aligned}\] 

\[ \displaystyle (y- c_1 ) ^2 + (x- c_2 ) ^2 = \lambda ^2 \left[ \frac{(x') ^2 }{(x') ^2 + (y ' ) ^2 } + \frac{(y') ^2 }{(x') ^2 + (y ' ) ^2 } \right]  \] 

\[ \displaystyle (y- c_1 ) ^2 + (x- c_2 ) ^2 = \lambda ^2 - \text{окружность}  \] 




%%-------------------------------%%

% Закрытие документа, если файл компилируется отдельно
\ifdefined\mainfile
    % Если это основной файл, не нужно заканчивать документ
\else
    \end{document}
\fi
% Условная компиляция для самостоятельной работы
\ifdefined\mainfile
    % Если это часть основного файла, не добавляем начало и конец документа
\else
    \documentclass[12pt, a4paper]{report}
    \usepackage{/Users/vladbelousov/Desktop/Semestr_4-FP-NSU/Настройка/library}
    \usepackage[utf8]{inputenc} % Подключение поддержки UTF-8
    \begin{document}
\fi

%%-------------------------------%%

\section{Вариационная задача на условный экстремум}

\[ \begin{cases}
    I [y_1, \ldots, y_n]  = \int_{t_0 }^{t_1 }F(t,y_1, \ldots, y_n, y_n ',..., y_n')dt \to  \mathrm{extz}\\
    y_i (t_0 ) = y_{i_0}, \quad y_i (t_1 ) = y_{i_1} , \text{ }  i =1, \ldots, n\\
    G(ty1, \ldots, y_n) = 0
\end{cases} \] 

Пример: Задача о геодезических на поверхностях

\begin{center}
    \includegraphics[width=0.5\textwidth]{/Users/vladbelousov/Desktop/Semestr_4-FP-NSU/ДфУ/Лекции_по_дням/image/19.png}
\end{center}

Найти кривую, соединяющую точки A и B, лежащие на поверхности, имеющую наименьшую длину. 

\[ \begin{cases}
x = x(t ) \\
y = y(t ) \quad  \text{уравнение кривой в параметрическом виде } ,\text{ }  t \in [t_0 , t_1]\\
z = z(t )
\end{cases} \] 

\[ I [x,y,z]  = \int_{t_0}^{t_1} \sqrt{(x' (t )) ^2 + (y' (t )) ^2 + (z' (t )) ^2 }dt \] 

\[\begin{cases}
x(t_0 ) = x_0 ,\quad x(t_1 ) = x_1 \\
y(t_0 ) = y_0 ,\quad y(t_1 ) = y_1 \\
z(t_0 ) = z_0 ,\quad z(t_1 ) = z_1
\end{cases} \] 

\textbf{Необходимое условие локального экстремума}:

Пусть \( \tilde{y_1 }, ..., \tilde{y }_n  \)  доставляют локальному экстремум для задачи (1). Тогда \( \exists    \lambda(t)   \) такая, что функции \( \tilde{y_1 }, ..., \tilde{y }_n  \) являются экстремалями вспомогательного функционала. 

\[ \tilde{I } [y_1, \ldots, y_n ] = \int_{t_0}^{t_1} (F + \lambda G(t))dt  \] 

\textbf{Без доказательства.} 

\section{Решение задачи о геодезических на сфере}

\begin{center}
    \includegraphics[width=0.4\textwidth]{/Users/vladbelousov/Desktop/Semestr_4-FP-NSU/ДфУ/Лекции_по_дням/image/20.png}
\end{center}

\[ x ^2  +y ^2 + z ^2 = R ^2  \] 

\[ I [x,y,z] = \int_{t_0}^{t_1} \sqrt{(x' (t )) ^2 + (y' (t )) ^2 + (z' (t )) ^2 }dt \] 

\[ \tilde{I } [x,y ,z ] = \int _{t_0}^{t_1}  \underbrace{\left(  \underbrace{\sqrt{(x' (t )) ^2 + (y' (t )) ^2 + (z' (t )) ^2 }}_{F}  + \lambda(t)(x ^2  +y ^2 + z ^2 - R ^2 ) \right)}_{\tilde{F}}  dt  \] 

\[\begin{cases}
    \displaystyle 2 x \lambda (t ) = \frac{d}{dt } \left( \frac{x' }{F}  \right) \quad (1)   \\
    \displaystyle 2 y \lambda (t ) = \frac{d}{dt } \left( \frac{y' }{F}  \right)  \quad (2)\\
    \displaystyle 2 z \lambda (t ) = \frac{d}{dt } \left( \frac{z' }{F}  \right) \quad (3)
\end{cases} \] 

\[ \begin{cases}
    \displaystyle (1) \cdot y + (2) \cdot (-x):\quad  y \frac{d}{dt } \left( \frac{x' }{F} \right) -x \frac{d}{dt } \left( \frac{y' }{F} \right) = 0 \\
    \displaystyle (2) \cdot z + (3 ) \cdot (- y) :\quad z \frac{d}{dt } \left( \frac{y' }{F} \right) -y \frac{d}{dt } \left( \frac{z' }{F} \right) = 0 \\
    \displaystyle (3) \cdot (-x) + (1) \cdot  z : \quad z \frac{d}{dt } \left( \frac{x' }{F} \right) -x \frac{d}{dt } \left( \frac{z' }{F} \right) = 0 
\end{cases} \] 

\[ \frac{d}{dt } \left( y \frac{x ' }{F }  - x \frac{y '}{F}  \right) = y ' \frac{x ' }{F } + y \frac{d}{dt } \left( \frac{x' }{F}  \right) - x ' \frac{y '}{F} - x \frac{d}{dt } \left( \frac{y' }{F}  \right) \] 

\[
\begin{aligned}
    \begin{cases}
        \displaystyle \frac{d}{dt } \left( y \frac{x' }{F } - x \frac{y ' }{F}  \right) = 0 \\
        \displaystyle \frac{d}{dt } \left( z \frac{y' }{F } - y \frac{z' }{F}  \right) = 0 \\
        \displaystyle \frac{d}{dt } \left( z \frac{x' }{F } - x \frac{z ' }{F}  \right) = 0 
    \end{cases} 
    \begin{cases}
        \displaystyle \frac{y x' - x y ' }{F } = c_1  \quad (1)\\
        \displaystyle \frac{z y' - y z ' }{F } = c_2  \quad (2)\\ 
        \displaystyle \frac{z x' - x z ' }{F } = c_3 \quad (3)
    \end{cases}
\end{aligned}\] 



\[ (1 )  \cdot z + (2)\cdot x + (3 ) \cdot (-y) : \]

\[ \frac{1}{F} \left[ z(y x' - x y ' ) + x (z y' - y z ' ) - y(z x' - x z ' ) \right] = c_1 z + c_2 x -  c_3 y  \] 

\[ \begin{cases}
    c_1 z + c_2 x -  c_3 y =  0 - \text{плоскость проходящая через (0,0,0)} \\
    x ^2 + y ^2 + z ^2 = R ^2 
\end{cases}  \] 

\begin{center}
    \includegraphics[width=0.3\textwidth]{/Users/vladbelousov/Desktop/Semestr_4-FP-NSU/ДфУ/Лекции_по_дням/image/21.png}
\end{center}

Геодезическая на сфере - дуга на большой окружности.

\chapter{Система малых колебаний}

\section{Линейные однородные системы малых колебаний}

\[ M \vec{x''}  + K \vec{x}  = 0  \] 

\[ \vec{x }  =\vec{x } (t) = \begin{pmatrix}
x_1( t)\\
\vdots\\
x_n (t)
\end{pmatrix} \quad  M =\begin{pmatrix}
m_{11} & ... & m_{1n}\\
\vdots &  & \vdots\\
m_{n_1} & ... & m_{nn} 
\end{pmatrix} \quad K=\begin{pmatrix}
    k_{11} & ... & k_{1n}\\
    \vdots &  & \vdots\\
    k_{n_1} & ... & k_{nn} 
\end{pmatrix}  \]

Пример: \( n =1 \Rightarrow m x'' + kx = 0 , \text{ } m >0 , \text{ } k > 0 \) 

\begin{center}
    \includegraphics[width=0.3\textwidth]{/Users/vladbelousov/Desktop/Semestr_4-FP-NSU/ДфУ/Лекции_по_дням/image/22.png}
\end{center}

\[ M -\text{ матрица масс} , \quad  K -\text{ матрица жесткостей } \] 

1) \( M = M^{\top} , \text{ }  K = K^{\top}   \text{ } (m_{ij}= m_{j i} ,\text{ } k_{ij} = k_{j i}   ) \) 

2) M > 0  (матрица положительна определена), \( K \geq 0 \) 

\begin{definition}
    Матрица \( M = M^{\top} \) называется положительно определенной, если \( \forall \vec{v } \in  \mathbb{R} ^{n } , \vec{v } \neq 0  \) выполняется \( (M\vec{v} , \vec{v} ) >0 \).  
\end{definition}

\textbf{Критерий Сильвестра: } \( M = M^{\top} > 0 \Leftrightarrow  \text{все главные миноры} > 0. \)

\textbf{1-ый способ:} Сведение к системе 1-го порядка. 

\[ \begin{aligned}
    \begin{cases}
        \vec{y_1 } = \vec{x } \\
        \vec{y_2 } = \vec{x '} 
    \end{cases}
    \begin{cases}
        \vec{y_1 '} = \vec{y_2} \\
        \vec{y_2  '} = \vec{x ''} = -M ^{-1} K \vec{x }  = - M ^{-1} K \vec{y_1}   
    \end{cases}
\end{aligned} \] 

\[ \frac{d}{dt} \begin{pmatrix}
\vec{y_1} \\
\vec{y_2} \\
\end{pmatrix} = \underbrace{\overbrace{\begin{pmatrix}
    0 &  & E\\
     &  & \\
    -M^{-1}K  &  & 0
    \end{pmatrix}}^{A}}_{n \times n}  \begin{pmatrix}
\vec{y_1} \\
\vec{y_2} \\
\end{pmatrix}   \] 





\[ \begin{pmatrix}
    \vec{y_1} \\
    \vec{y_2} \\
\end{pmatrix} =\underbrace{e^{t A }}_{(2n \times  2n)} \underbrace{\vec{c}}_{(2n \times  1)}  = \begin{pmatrix}
\Phi_{11}(t) & \Phi_{12}(t)\\
\Phi_{12}(t) & \Phi_{22}(t)
\end{pmatrix} \begin{pmatrix}
\vec{c_1} \\
\vec{c_2} 
\end{pmatrix}  \]

\[ \vec{x} (t ) = \vec{y_1 } (t ) = F_{11} (t ) \vec{c_1 }  + F_{12} (t ) \vec{c_2 } \quad  (\text{2n констант} )    \] 

\begin{lemma}
    Если \( M = M^{\top} > 0  \), то \( \exists  M^{-1}  \)  
\end{lemma}

\begin{proof}
    \[  \] 
    Пусть не существует \( M^{-1} \Rightarrow \kern-30pt\underbrace{\mathrm{det } M = 0}_{\displaystyle \underset{ \displaystyle \det (M - 0 E) = 0}{\Rightarrow \lambda = 0 - \text{собств.знач.} }}\kern-30pt \Rightarrow \exists  \vec{v } \neq 0 : M \vec{v } = 0  \) 

    \[ (M\vec{v } , \vec{v }     ) = (0, \vec{v } ) = 0 - \text{  противоречие}  \] 
\end{proof}


\begin{proposition}[из алгебры]
    Пусть \( A = A^{\top} \Rightarrow  \)  все собственные числа \( \lambda_j \in  \mathbb{R}.  \) 

    Пусть  \( A = A^{\top} > 0 \Rightarrow  \) все собственные числа \( \lambda_j > 0.  \)
\end{proposition}

\begin{proposition}[из алгебры]
    Пусть \( A = A^{\top} \Rightarrow   \) в \( \mathbb{R} ^n  \)  существует базис из собственных векторов, то есть нет присоединенных 
\end{proposition}

\begin{proposition}[из алгебры]
    Пусть \( A = A^{\top} \Rightarrow A = U D U^{-1}  \), \( \begin{pmatrix}
    \lambda_1 &  & 0\\
    &\ddots  & \\
    0& & \lambda_n
    \end{pmatrix} \) 

\( U  \) - ортогональная  матрица, то есть \( U ^{-1} = U^{\top }   \)  
    
\end{proposition}

\textbf{2-ой способ:}

\begin{definition}
    Число \( \lambda     \)  называется собственным числом системы (1), если
    
    \( \det (K -\lambda M ) = 0 \) 
\end{definition}

\begin{definition}
    Вектор \( \vec{v }  \neq \vec{0}  \)  называется собственным вектором системы (1) (вектором нормальных колебаний), если \( (K -\lambda M ) \vec{v } = 0 \) 
\end{definition}

\begin{theorem}
    Существует n собственных чисел системы (1) и \( \lambda_{i } \geq 0 , \forall  i = 1,2,...,n \) 
\end{theorem}

\begin{proof}
\[  \]

1) \( \det (K -\lambda M )  = 0 \) 

\[ \det (M(M^{-1} K - \lambda E )) = 0 \] 

\[ \underbrace{\det M}_{ \neq 0} \det (M^{-1 } K - \lambda E ) = 0 \Rightarrow \text{ существует n собственных чисел}  \] 

2) \( \vec{v_j}  \)  - собственные вектора \( \Rightarrow K \vec{v_j} = \lambda_j M \vec{v_j} \text{ } | \cdot \vec{v_j}   \)  

\[ \underbrace{(K v_{j }  , v_j)}_{\geq 0}  = \lambda_j \underbrace{(M v_j ,v_j)}_{>0} \Rightarrow \lambda_j \geq 0  \] 
\end{proof}

\begin{theorem}
    В \( \mathbb{R} \)  существует базис из собственных векторов системы (1).
\end{theorem}

\textbf{Доказательство будет позже.} 

\begin{theorem}
    Пусть  \( M = M^{\top} > 0 , K =K ^{\top} \geq 0 , \lambda_1, \ldots, \lambda_n \geq 0 \)  - собственные числа системы (1), \( \vec{v_1 },..., \vec{v_n}   \)  - собственные вектора системы (1), соответвующие числам \( \lambda_1, \ldots, \lambda_n \). Тогда все решения системы (1)  имеют вид: 

    \[ x(t) = \sum_{j=1} ^{n } q_j (t)\vec{v_j},   \] 

    где \( q_j(t) \)  - решение дифференциального уравнения: \( q_j '' + \lambda_j q_j = 0 \) 
\end{theorem}

\begin{proof}
    \[  \] 
    По теореме 2 \( \vec{v_1},..., \vec{v_n }   \)  - базис в \( \mathbb{R} ^ n \). При фиксированном t \( x(t ) \in  \mathbb{R}^{n} \Rightarrow \vec{x } (t)  \)  раскладывается по базису: \( \vec{x } (t) = \sum_{j=1} ^{n } q_j (t)\vec{v_j}  \) 

    Подставляем \( x(t) \)  в систему (1): 

    \[ M \sum_{j =1}^n q_j '' (t ) \vec{v_j} + K \sum_{j =1}^n q_j (t ) \vec{v_j} = 0  \] 

    \[ \sum_{j=1}^ n \left( q_j '' (t ) M \vec{v_j } + q_j (t )\underbrace{ K \vec{v_j } }_{\lambda_j M \vec{v_j } } \right) =0\] 

    \[ \sum _{j=1}^ n \left( q_j '' (t ) M \vec{v_j } + \lambda_j q_j (t )M\vec{v_j }  \right) = 0 \text{ }  | \cdot  M^{-1}  \] 

    \[ \sum _{j=1}^ n \left( q_j '' (t  ) + \lambda_j q_j (t ) \right) \vec{v_j }= 0 , \text{ }  \forall  t \in  \mathbb{R}   \] 

    \[ \text{Т.к} \vec{v_1 },..., \vec{v_n} \text{ линейно независимы, то }  q_j '' (t ) + \lambda_j q_j (t) =0    \] 


\end{proof}

\begin{center}
    \textbf{Замечание.} \( q_j ''(t ) + \lambda_j q_j (t) = 0 \) 
    1) \( \lambda_j = 0 \Rightarrow q_j (t ) = c_1 t + c_2  \) 

    2) \( \lambda_j > 0 \Rightarrow q_j (t )  = c_1 \cos (\sqrt{\lambda_j} t) + c_2 \sin (\sqrt{\lambda_j} t) \)
\end{center}

\begin{definition}
    \( \omega_1 = \sqrt{\lambda_1},..., \omega_n = \sqrt{\lambda_n} \) называется собственными частотами колебаний системы (1).
\end{definition}

%%-------------------------------%%

% Закрытие документа, если файл компилируется отдельно
\ifdefined\mainfile
    % Если это основной файл, не нужно заканчивать документ
\else
    \end{document}
\fi
%Март
% Условная компиляция для самостоятельной работы
\ifdefined\mainfile
    % Если это часть основного файла, не добавляем начало и конец документа
\else
    \documentclass[12pt, a4paper]{report}
    \usepackage{/Users/vladbelousov/Desktop/Semestr_4-FP-NSU/Настройка/library}
    \usepackage[utf8]{inputenc} % Подключение поддержки UTF-8
    \begin{document}
\fi

%%-------------------------------%%

\begin{proof}[Докозательство теоремы 2.]
    \[  \] 
    \[ M = M ^{ \top } > 0 \Rightarrow \underset{>0}{\lambda_1 (M )},..., \underset{>0}{\lambda_n (M )} - \text{ собственные числа матрицы }  M \] 

    \[ M = U \begin{pmatrix}
    \lambda_1 (M) &  & 0\\
     & \ddots & \\
    0 &  & \lambda_n(M)
    \end{pmatrix} U^{-1} , \text{ можно взять } U \text{ - ортогональную матрицу, то есть } U^{ -1} = U^{\top}    \] 


\[ \sqrt{M} = U \begin{pmatrix}
    \sqrt{\lambda_1 (M)} &  & 0\\
     & \ddots & \\
    0 &  & \sqrt{\lambda_n(M)}
    \end{pmatrix}  U^{-1} \] 

    \[ \text{Видно, что: } \sqrt{M} \sqrt{M}  =M\]

    Пусть \( \vec{v_j}   \) - собственный вектор: \( (K- \lambda_j M ) \vec{v_j } =\vec{0}   \) 

    \[ (K - \lambda_j \sqrt{M } E \sqrt{M }) \vec{v_j} = 0 \] 

    \[ \sqrt{M } (\underbrace{(\sqrt{M })^{-1} K (\sqrt{M})^{-1}}_{A} - \lambda_j E   ) \sqrt{M } \vec{v_j} = 0 \] 

    \[ \lambda_j - \text{ собственное число } A , \text{ }  \sqrt{ M } \vec{v_j } -\text{ собственный вектор  }  A \] 

    \[ A= A^{\top} , \quad  A^{\top } =\underbrace{ [(\sqrt{M } )^{-1} ]^{\top }}_{(\sqrt{M })^{-1} } \underbrace{K^{\top }}_{K} \underbrace{ [(\sqrt{M } )^{-1} ]^{\top }}_{(\sqrt{M })^{-1} }    \] 

    Из алгебры (утверждение 2.) в \( \mathbb{R}     ^{ n }  \) существует базис из собственных векторов матрицы \( A : \sqrt{M }\vec{v_1 },..., \sqrt{M } \vec{v_n}    \). Так как \( \det \sqrt{M }<0 ,  \)  то \( v_1, \ldots, v_n \) - базис \( \mathbb{R} ^n  \). 
    \[  \] 
\end{proof}
 

\section{Линейные неоднородные системы  малых колебаний}

\[ M\vec{x} ''+ K\vec{x}  = \vec{f } (t)  \quad  (1 )\] 

\[ \vec{x }  =\vec{x } (t) = \begin{pmatrix}
x_1( t)\\
\vdots\\
x_n (t)
\end{pmatrix} , \text{ }  M, K - (n \times n) , \text{ }  M= M^{\top } > 0 , K =K ^{\top } \geq  0 , \vec{f } (t) = \begin{pmatrix}
f_1(t)\\
\vdots\\
f_n(t)
\end{pmatrix}\] 

1-способ. Сведение к системы 1-го порядка 

2-способ. 

\begin{theorem}
    Пусть \( \lambda_1, \ldots, \lambda_n \) - собственные числа, то есть \( \det (K - \lambda_j M ) = 0 \), \( v_1, \ldots, v_n  \)  - собственные вектора, то есть \( (K - \lambda_j M )v _j = 0 \) 

    Пусть \( \lambda_1 \neq \lambda_2  \). Тогда \(\kern-0.8cm \underbrace{(M v_1 ,v_2 )}_{v_1,v_2 - M  \text{ - ортогональны} } =  \underbrace{(K v_1 ,v_2 )}_{v_1,v_2 - K \text{ - ортогональны} }\kern-0.8cm  = 0 \)  

\end{theorem} 

\begin{proof}
    \[  \] 

    \[ \begin{aligned}
        \begin{cases}
            K v_1 = \lambda_1 M v_1  | \cdot v_2 \\ 
            K v_2 = \lambda_2 M v_2  | \cdot v_1 \\ 
        \end{cases} 
        \begin{cases}
        (K v_1 , v_2 ) = \lambda_1 ( M v_1 , v_2 ) \\ 
        (K v_2, v_1 ) = \lambda_2 ( M v_2 , v_1 ) \\ 
        \end{cases}
    \end{aligned}\] 

    \[ (K \vec{v } _1 , \vec{v }_2   )  = (v_1, K^{\top } v_2) = (v_1, K v_2 ) = (K v_2 , v_1 )\] 

Вычитаем одно из другого: 

\[ 0 = \lambda_1 (M v_1 ,v_2 ) - \lambda_2 (M v_2 , v_1  ) = \underbrace{(\lambda_1 - \lambda_2 )}_{\neq 0} (M v_1 ,v_2 ) \Rightarrow (Mv1, v_2 ) = 0 \Rightarrow (K v_1 ,v_2 ) = 0 \] 

\end{proof}

\begin{theorem}
    Пусть \( \lambda_1=, \ldots, =\lambda_p \) - собственное число кратности \( p \). Тогда существует собственные вектора \( \vec{w_1} , \ldots,\vec{w } _{p}   \), которые являются \( M  \)   - ортогональными, то есть \( (M w_i, w_j ) = 0  \) при \( i \neq j \) 
\end{theorem}


\begin{proof}

\[  \] 
    Из параграфа 1 (теорема 2) мы знаем, что \( \exists  \vec{v } _1 ,..., \vec{v } _p  \) - линейно независимые собственные вектора. 
    
    Метод \( M \) - ортогонализации Грама-Шмидта: 

    \[\vec{w_1 } = \vec{v_1 }    \] 

    \[ \vec{w_2 } = \vec{v_2 } + \alpha \vec{v_1 } , \text{ }  \alpha - ? , \text{  } ( M \vec{w_2 }, \vec{w_1 }  ) = 0    \] 

    \[ \underbrace{(M \vec{w_2 }, \vec{w_1}  ) }_{0}= (M\vec{v_2 }, \vec{w_1 }  ) + \alpha (M \underset{=\vec{w_1}}{\vec{v_1 }}, \vec{w_1}   )  \Rightarrow \alpha =- \frac{ (M \vec{v_2 }, \vec{w_1}  )}{(M \vec{w_1 }, \vec{w_1}  )} \] 

    Пусть \( \vec{w_1 },..., \vec{w_{m-1} }    \) построены, причем \( (M \vec{w_i } , \vec{w_j}  )= 0 ,\text{ } i \neq j , \text{ } i,j= 1, \ldots, m-1 \) 

    \[ \vec{w_m } = \vec{v_m }+ \sum_{j =1}^{m -1 }  \beta_j \vec{w_j } , \text{ } \beta_j -? , \text{ } (M \vec{w_m }, \vec{w_i}  ) = 0, \text{ } i=1, \ldots, m-1    \]
    
    \[ \underbrace{(M \vec{w_m } , \vec{w_i }  )}_{0} = (M \vec{v_m }, \vec{w_i} ) +\underbrace{ \sum_{j =1}^{m-1 } \beta_j (M\vec{w_j }, \vec{w_i}  )}_{\beta_i \underbrace{(M\vec{w_i }, \vec{w_i }  )}_{>0}}  \] 

    \[ \Rightarrow \beta_i = \frac{ - (M \vec{v_m }, \vec{w_i }  )}{(M \vec{w_i }, \vec{w_i}  )}  ,\text{ }  j= 1, \ldots, m-1  \Rightarrow \vec{w_1 },.., \vec{w_p} \text{ - M-ортогональны}   \] 
\end{proof}

\begin{theorem}
    Пусть \( \lambda_1, \ldots, \lambda_n \)  - собственные числа системы (1), \( \vec{w_1 }, ,..., \vec{w_n}   \) - собственные вектора, которые М-ортогональны. Тогда решение (1) имеет вид: 

    \[ \vec{x } (t ) = \sum_{j =1}^{n } q_j \vec{w_j }  \] 

    \(\text{, где }  q_j (t ) \text{ - решение дифференциального уравнения: }  q_j '' + \lambda_j q_j = \tilde{f_j }(t) \) 

    \[ \tilde{f }_j = \frac{(\vec{f } (t ), \vec{w_j} )}{(M \vec{w_j} , \vec{w_j} )} \] 
\end{theorem}

\begin{proof}
\[  \] 

    Так как \( \vec{w_1 },..., \vec{w_n }    \)  - базис в \( \displaystyle \mathbb{R}    ^n , \text{ } \vec{x } ( t  ) \in \mathbb{R} ^ n \Rightarrow \vec{x } (t ) = \sum_{j =1}^ n q_j ( t ) \vec{w_j } - \text{ решение (1)}. \) 

    \[ M \sum_{j =1}^ n q_j '' (t ) \vec{w_j } + \underbrace{K \sum_{j =1}^ n q_j (t ) \vec{w_j }}_{K \vec{w_j }= \lambda_j M \vec{w_j}  } = \vec{f } (t )   \] 

    \[ \sum_{j =1}^ n (q_j ''(t )+ \lambda_j q_j (t ))M \vec{w_j } = \vec{f } (t )  | \cdot \vec{w_i}   \] 

    \[ \sum_{j =1}^ n (q_j '' (t )+ \lambda_j q_j (t ))\underbrace{(M\vec{w_j }, \vec{w_i }  )}_{\tiny\begin{aligned}
    =0 , \text{ если } j\neq i \\
    \neq 0, \text{ если } j = i  
    \end{aligned}} = (\vec{f } (t ), \vec{w_i} ) \] 

    \[ (q_i '' (t ) + \lambda_i q_i (t ))(M \vec{w_i},  \vec{w_i }  )= (\vec{f } (t) , \vec{w_i} ) \] 

\end{proof}

\chapter{Зависимость решения от параметров}

\section{Непрерывная зависимость решений от параметров и начальных данных}

\[ \begin{cases}
    y ' =  f (t,y ) , \quad  f : \mathbb{D} \to  \mathbb{R} , \text{ } \mathbb{D} \subset \mathbb{R} ^2 , \text{ } \mathbb{D} \text{ - решение открытое}  \\ 
    y(t_0) = y_0
\end{cases} \] 


\begin{theorem}[Теорама Пикара]
    Если \( f \in  C(\mathbb{D} ) , \text{  } \exists  \frac{\partial  f }{\partial  y } \in  C(\mathbb{D} ) \Rightarrow \forall  (t_0, y_0 ) \in  D \text{ } \exists !    \)  непродолжаемое  решение задачи Коши, определенной на открытом интервале \( (\alpha, \omega) \) 
\end{theorem}

Будем менять \( y_0 \)

Решение задачи Коши: \( y (t; y_0) \) 

\begin{center}
    \includegraphics[width=0.5\textwidth]{/Users/vladbelousov/Desktop/Semestr_4-FP-NSU/ДфУ/Лекции_по_дням/image/23.png}
\end{center}

Вопрос: если \( y_0 \approx y_0^* \), можно ли утверждать, что \( y(t, y_0 ^* ) \approx y (t, y_0) \) 

Пример: 

\[ \begin{aligned}
    \begin{aligned}
        &\begin{cases}
            y ' = y ^2  \\ 
            y(0 ) = y^* = 0 
        \end{cases} \\
        &y(t, 0 )=0 , \text{ } t \in (-\infty , +\infty )
    \end{aligned}
    \quad \quad  
    \begin{aligned}
        &\begin{cases}
            y ' = y ^2 \\
            y(0) = y_0 > 0
        \end{cases} \\
        &y(t,y_0) = \frac{1 }{\frac{1}{y_0 } -t } , \text{ } t \in  \left( -\infty , \frac{1}{y_0}  \right)
    \end{aligned}
\end{aligned} \] 

\begin{center}
    \includegraphics[width=0.4\textwidth]{/Users/vladbelousov/Desktop/Semestr_4-FP-NSU/ДфУ/Лекции_по_дням/image/25.pdf}
\end{center}

\begin{theorem}
    Пусть \( f \in C(\mathbb{D}) , \text{ } \exists \frac{\partial f}{\partial y} \in C(\mathbb{D}) \). Пусть \((t_0, y_0^*) \in \mathbb{D}\). Пусть \( y(t, y_0^*) \) - решение задачи Коши, определенное на интервале \( (\alpha, \omega) \). Возьмем \( [t_1, t_2] \subset (\alpha, \omega) \). Тогда:

    1) \( \exists \Delta > 0 , \text{ }  \forall  y_0 : \left\lvert y_0 - y_0 ^*  \right\rvert < \Delta \Rightarrow y(t, y_0)\)  определенно при \( t \in  [t_1, t_2 ] \); 
    
    2) \( y(t, y_0 ) \xrightarrow{y_0 \to  y_0^* } y(t, y_0^*) , \text{ }  t \in [t_1, t_2 ]  \)
\end{theorem}

Пример: 

\[ (t_0, y_0 ^* ) = (0,0 ) \Rightarrow y (t, y_0 ^* ) \equiv 0 , \text{ } (\alpha , \omega ) = (-\infty , + \infty  ). \text{ Возьмем: } [-T, T]  \] 

\begin{center}
    \includegraphics[width=0.4\textwidth]{/Users/vladbelousov/Desktop/Semestr_4-FP-NSU/ДфУ/Лекции_по_дням/image/24.png}
\end{center} 
\[ 1) \text{ } y_0 > 0 ;\text{ }  T < \frac{1}{y_0 } \Leftrightarrow y_0 < \frac{1}{T } =\Delta  \] 

\[ 2) \text{ } y(t,y_0 )= \frac{1}{\frac{1}{y_0 } - t    } \xrightarrow{y_0 \to  0 } 0 , \text{  }  t \in [-T,T]    \] 

%%-------------------------------%%

% Закрытие документа, если файл компилируется отдельно
\ifdefined\mainfile
    % Если это основной файл, не нужно заканчивать документ
\else
    \end{document}
\fi

\vfill
\begin{center}
    \textbf{Пролетарии всех стран, соединяйтесь!}
\end{center}
\vfil

\end{document}